\usepackage[T1]{fontenc}
\usepackage[utf8]{inputenc}
\usepackage{kpfonts}

\usepackage{textgreek}

\usepackage[final]{graphicx}

%% remove colorlinks option when ready for print
\usepackage[final,hyperindex,hyperfootnotes,bookmarksnumbered,colorlinks]{hyperref}

\usepackage{amsmath}

\newsubfloat{figure} % allow subfigures

\usepackage[textsize=footnotesize]{todonotes}
%% new command for box about missing references
\newcommand{\addref}[1]{\todo[color=red!40,size=\tiny]{Add reference: #1}}

\usepackage{tikz}

\usepackage[sectionbib,round,comma]{natbib}
\bibliographystyle{agu}

\usepackage{siunitx}
\DeclareSIUnit{\bp}{bp} % base pairs
\newcommand{\pcent}[1]{\SI{#1}{\percent}}
\newcommand{\dc}[1]{\SI{#1}{\degreeCelsius}}

\newcommand{\captionIntro}[2]{\caption[#1]{\textbf{#1} #2}}

%% Just like we have cite and citep to cite in text and between parentheses,
%% have the same for fref, tref, etc...
\newcommand{\frefp}[1]{(\fref{#1})}
\newcommand{\Srefp}[1]{(\Sref{#1})}

\newcommand{\species}[1]{\textit{#1}}
\newcommand{\command}[1]{\texttt{#1}}

%% NCBI Style Guide, Chapter 5 "Style Points and Conventions", recommends
%% italic for gene names (except in long list of genes), and roman for
%% protein names.
\newcommand{\gene}[1]{\textit{#1}}
\newcommand{\protein}[1]{#1}

\newcommand{\Kon}{$K_{on}$}
\newcommand{\Koff}{$K_{off}$}
\newcommand{\halflife}[1][]{$t_{1/2}$#1}

\newcommand{\G}[1]{G$_#1$}  % for G0, G1, and G2 phases
