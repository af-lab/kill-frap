\section{Discussion}

    We wished to quantitatively determine the effect on chromatin stability
    of SIN mutations in core histones H3 and H4 known 
    to be destabilising \textit{in vitro} and to affect cell growth in \textit{S. cerevisiae}.
    We set out to use a previously reported model for circle FRAP
    which accounts for multiple errors in a typical FRAP modelling \addref{?}.
    However, FRAP experiments with core histones involve highly extended imaging timeframes
    because recovery is incomplete even after 8~hours \citep{KimuraCook}.
    This led us to address a series of technical challenges in
    collecting valid quantiative recovery data over extended time periods.

% Cell movement

    The first problem faced was cell movement.
    This is an expected property of actively dividing cells, 
    and those that are unable to move are likely to be dead.
    Moderate amounts of simple translational and rotational movement 
    around the $z$ axis can still be tracked in a single focal plane.

    We attempted to reduce motion by taking advantage of 
    the fact that many primary cells move into the quiescent \G0{}~phase 
    of the cell cycle when they reach high densities. 
    This contact inhibition is a natural mechanism that controls their cellular growth in 
    multicellular organisms, and in tissue culture it results in
    a stop in proliferation with formation of a monolayer of healthy cells.

    However, levaraging of contact inhibition increases cell handling and reduces transfection efficiency,
    and means that stable cell lines based on immortalised cells cannot be used. 
    The potential inability to compare results with published data for 
    immortalised cell lines such as HeLa is disadvantageous.

    Despite achieving a monolayer of healthy cells that could be maintained stably over 2 weeks,
    individual transfected primary horse fibroblasts still showed some movement
    although they otherwise exhibited characteristics of contact inhibition.
    Furthermore, nuclei in these cells displayed a helical motion on the direction of cell movement \frefp{fig:kill-frap:confluent-horse}.

    The possibility of chemically inhibiting cells to reduce motion was also considered.
    Previous FRAP experiments with core histones were performed using multiple inhibitors of protein synthesis \citep{KimuraCook}, 
    but these studies revealed inhibitor-dependent variations in kinetics 
    and the authors qualified their conclusions about the absolute accuracy of exchange parameters measured.

    To address the problem of cell movement we instead developed a computational approach
    by writing a program for cell tracking by normalised cross-correlation template matching  \addref{?}.
    Using automated analysis enabled us to process the large numbers of cell images 
    required to provide statistically valid quantitative measurements of core histone exchange.

% Compositional changes

    The second challenge to measuring core histone exchage by FRAP is that
    the basis of FRAP is a chemical equilibrium between formation of a complex 
    with freely diffusing proteins and vacant static binding sites.
    Achieving absolute equilibrium is unlikely in the dynamic cell environment
    undergoing complex transcriptional and translation responses.
    In particular, the large chromatin compositional and structural changes during the cell cycle
    present sepcific challenges to measuring core histone exchange over extended time periods.

    DNA replication involving polymerase passage and repackaging of the duplicated genome
    in S~phase clearly unbalances the equilibrium.
    Chromosome compaction in mitosis generates a chromatin environment that is distinct from interphase.
    This limits FRAP experiment to either \G1{} or \G2{} phases.
    The HeLa cell cycle has a typical \G1{} phase of 11.7~hours and a \G2{} phase of 3~hours \citep{HeLaCellCycle}
    so the extended time periods for core histone measurements require FRAP experiments
    to a start in early \G1{} \frefp{fig:kill-frap:cell-cycle}.

    Post-mitotic chromosomes take some time to migrate within the nucleus 
    and rebuild the interphase nuclear architecture during early \G1{}.
    This process that has been estimated to take approximately 2 hours
    \citep{visualizationG1chromosomes,earlyg1position,RelativeChromosomePosition}.
    This defines the window for extended FRAP experiments from approximately 3 to 11 hours after anaphase
    in HeLa cells, although cells lines with even longer \G1{} phase are known \citep{PancreaticCells}.

    We wished to avoid the use of drugs for cell cycle arrest since this has been
    shown to influence FRAP results. 
    We also considered serum starvation to move cells into the
    quiescent \G0{} phase since this could affect the relevance of measurong core histone exchange \citep{SerumStarvation} \addref{?}.

      \begin{figure}
        \centering
        %% based on original code from Robert Vollmert
        %% http://www.texample.net/tikz/examples/pie-chart/
        \newcommand{\slice}[4]{
          \pgfmathparse{0.5*#1+0.5*#2}
          \let\midangle\pgfmathresult

          % slice
          \draw[thick,fill=black!10] (0,0) -- (#1:1) arc (#1:#2:1) -- cycle;

          % outer label
          \node[label=\midangle:#4] at (\midangle:1) {};

          % inner label
          \pgfmathparse{min((#2-#1-10)/110*(-0.3),0)}
          \let\temp\pgfmathresult
          \pgfmathparse{max(\temp,-0.5) + 0.8}
          \let\innerpos\pgfmathresult
          \node at (\midangle:\innerpos) {#3};
        }
        \begin{tikzpicture}[scale=3]
          \newcounter{a}
          \newcounter{b}
          %% Total cell cycle is 24.5 hours, G1 is 11.7h, S is 8.8h,
          %% G2 is 3h, M is 1h. The problem is that the counters can't handle
          %% decimal places so we have a variable with the actual time for
          %% the text, and another one times 10 to calculate the angle.
          \foreach \p/\t/\l in {117/11.7/\G1, 9/0.9/M,
                                31/3.1/\G2, 88/8.8/S}
            {
              \setcounter{a}{\value{b}}
              \addtocounter{b}{\p}
              \slice{36*\thea/24.5} % we multiply by 36 instead of 360 because
                    {36*\theb/24.5} % the time is already times 10
                    {\l}{\t{} hours}
            }
        \end{tikzpicture}
        \captionIntro{HeLa cell cycle phases and timing}
          {
            Under optimal growth conditions, the HeLa cell has a median
            doubling time of approximately 24.5~hours, with \G1~and
            S~phases alone having a length of 11.7 and 8.8 hours each \citep{HeLaCellCycle}.
            Since the FRAP experiment must avoid the S phase and last for
            at least 8~hours, the only possibility while using cells in normal
            growth conditions is to start the experiment in the early \G1~phase.
          }
        \label{fig:kill-frap:cell-cycle}
      \end{figure}

    Instead, we devloped a procedure to track cells manually during mitosis 
    where visual identification of the cell cycle is possible. 
    This allowed us to minimise interventions and effects on the cell normal growth
    while identifying individual cells exactly 3 hours after start of \G1{}.
    This has the added advantage of allowing time for maturation of GFP 
    expressed during the establishment of interphase.
    The time interval between images was increased and 
    both resolution and laser intensity reduced
    to minimise any phototoxicity or bleaching arising from the extra imaging required.
    This resulted in a set of selected early \G1{} cells suitable for FRAP experiments.

    The fluorescently tagged histone proteins are constitutively expressed 
    under the control of an EF-1\textalpha{} promoter, 
    and lack the 3' regulatory features of the native H2B gene transcripts.
    This regulation does not follow the normal expression program of a histone gene 
    and hence could alter the distribution of the histone in chromatin.
    Constant expression of tagged histones by a strong constitutive promoter
    will enrich them in the \G1{} and early S~phase pools
    making subsequent incorporation in euchromatin more likely,
    relative to mid-late S~phase where heterochromatic sequences are repliacted and packaged \citep{DNA-replication-timing}.

    A more realistic tagged histone expression profile can be achieved using 
    flanking regulatory regions from native histone genes,
    as demonstrated for H3 and CENP--A \citep{pMH3-plasmid,Kevin-pCA-TAG}.
    Another potential solution is to insert GFP in-frame into the native gene locus by genome engineering,
    although the redundancy between 18 canonical H2B genes means
    that identifying the most appropriate isoform to target could introduce complexities.

    Protein synthesis inhibitors were used by \cite{KimuraCook} to address this issue,
    but this has the disadvantage of also potentially affecting many other processes as discussed above.

      %% TODO: would be cool to create this figure
%      \begin{figure}
%        \centering
%        \missingfigure{a schematic of cell cycle, soluble pool}
%        \captionIntro{Distribution of tagged and endogenous histones during cell cycle}
%                     {
%                       This would be at least 3 different subplots. The first
%                       and the second are like the ones in Fig 7A of Kimura and
%                       Cook paper. The third one would show the ratio of each
%                       histone over time, i.e., 100\% tagged during all cell
%                       cell cycle and some endogenous during S phase. In this
%                       plots, also note where euchromatin and heterochromatin
%                       are replicated.
%                     }.
%        \label{fig:kill-frap:messy-histone-expression}
%      \end{figure}

% Movement of the reference

    The final challenge to measuring core histone exchange by FRAP that we identified 
    was non-homogenous regional movement of chromatin itself.
    This undermines the assumption of FRAP analysis that binding sites 
    remain immobile throughout the FRAP experiment.
    This assumption is required to interpret recovery
    as the rate of movement of freely diffusing unbleached molecules into the
    bleached area which allows the kinetic rates \Kon{} and \Koff{} to be estimated.
    If the chromatin binding sites move then the recovery curve becomes a 
    much more complex function of both binding site movement and free diffusion.

    The chromatin movement is recognisable both 
    by changes in the intra-nuclear features of the fluorescent chromatin 
    and by changes in the circular bleach spot.
    Although some of these effects are subtle when observed by photobleaching,
    the circle photoactivation by inverse FRAP demonstrates 
    clear non-homogenous reshaping of chromatin.
    Equivalent chromatin movement has also been reported 
    for H4--PAGFP in strip photoactivation \cite{H4PAGFP-chromatin-movement}.

    The movements we observed were in the range of \todo{what sort of dimensions} 
    and exhibited complicated shapes consistent with channelling.
    This is consistent with chromosome distribution in nuclei that is 
    territorial on the scale of 5 \textmu m \addref{PMID:10866946}, 
    with interchromosomal channels of 10-100 nm \addref{PMID:16317046}
    
    The clarity of H2B--GFP imaging in inverse FRAP suggests the opportunity to analyse the 
    properties of pathways taken by diffusing core histones. 
    For example, simultaneous use of combined photoactivation and photobleaching 
    of complementary dimer and tetramer histones could 
    enable relative diffusion rates and paths to be determined.
    Alternatively, an enzymatic mechanism to incorporate a 
    complementary photo-differented label into DNA  would facilitate masking for 
    the original location at the same time as tracking the histone diffusion
    and enable quantitative FRAP.
    Nevertheless, it is important to recognise that such experiments would test the 
    resolution and sensitivity limits of microscopes.

\section{Conclusion}

    FRAP has been continuously improving with increased capabilities of light microscopy
    and subtle kinetic models now able to take into account
    an increasing number of biophysical features such as container size, 
    non-homogeneous distribution of fluorescence and profile of bleach spot.

    Despite these advances, the ability to perform FRAP experiments 
    over extended time periods of several hours for highly stable complexes
    such as chromatin is limited by the dynamic nature of the cell.

    We overcame the challenges of cell mobility and selection of cells in \G1{} phase,
    but were unable to develop a method to adjust for changes in chromatin structure within the cell nucleus.
    While a photobleached spot appears stable and can be tracked over several hours, 
    small natural disturbances impact on the recovery and estimation of kinetic parameters.
    We find that FRAP is suitable for semi-quantitative estimates of slowly diffusing molecules
    but not for the precise quantitative comparisons required to compare core histone mutations.


%  Single-molecule imaging of histones for short period of times in live cells
%  has recently been reported using super-resolution imaging\addref[nature methods 7(9):717-719,
%  2010 and nature methods 8(1):7-9, 2011].

%  Also, use of PAGFP has been used to measure dynamics of H4 over \SI{90}{\ms} reporting
%  differences between interphase chromatin and mitotic chromosomes\addref[Saera Hihara et al 2012].
%  However, the difference between these two phases is the highest and might not be comparable to
%  the difference between histones variants\todo{study this. Someone must have measured this}.
