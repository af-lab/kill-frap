\section{Materials and Methods}

  Cells were transformed by lipofection \Srefp{methods:lipofection}.
  For the generation of stable cell lines,
  cells were split to a low confluence and treated
  with \SI{3}{\ug\per\ml} Blasticidin-S.
  Colonies were screened by fluorescence microscopy
  and positive individual colonies were selected for growth
  and fluorescence activated cell sorting (FACS) to generate
  homogeneous highly fluorescing cell lines.
  Transfection was performed 48h before imaging of
  transient expressing cells.

  Primary horse fibrolasts were a gift from Prof.\@ Elena Giulotto
  (University of Pavia). HeLa cells were ATCC line CCL-2.

  Confocal microscopy was performed in a Zeiss LSM510 Meta microscope
  using glass bottom LabTek~II chambers.  Wide-field fluorescence
  microscopy was performed with an Applied Precision DeltaVision Core
  system using \SI{35}{\mm} glass bottom MatTek dishes.  In both
  cases, imaging was performed within an acrylic environmental chamber
  at a temperature of \dc{37} and \pcent{5} CO$_2$.

  %% See the CropReg.m script for more details.  For example, only the
  %% region surrounding the original position was used for performance
  %% and robustness.

  Cell movement between time frames was calculated by consecutive
  template-based registration using normalised cross-correlation.
  The CropReg script we developed for this purpose is available 
  as free open source software. Nuclei of interest were identified 
  on the first time frame and used as templates on the subsequent images.
  To correct for rotational movement
  around the $z$ dimension of the optical axis system,
  registered frames were aligned by rigid body geometric transformation
  in the ImageJ \citep{imagej1} plugin StackReg \citep{stackreg}.

  Automatic extraction and processing of FRAP recovery curves was
  performed with the GNU Octave programming language \citep{octave}
  and the Octave Forge image package.  Source code written in Matlab
  for a previously reported circle FRAP model \citep{mcnally-frap-code}
  was kindly gifted by the original authors and ported to GNU Octave.
  We developed \command{frapinator}, a new program written in GNU Octave 
  to automate analysis with multiple commandline options
  and released it in a new FRAP Octave package as free open source software.
  It includes all individual functions for image
  pre-processing and FRAP fitting \Srefp{sec:software:octave-frap}.
