\section{Conclusions: why histone dynamics is not measurable by FRAP}
  
  I have attempted to establish FRAP as a method for measurement of histone variants. Such
  an experiment requires observation over an extremely long time interval, challenging several
  assumptions of a typical FRAP experiment.
  
  Workarounds were found to each of them, without the introduction of more factors or decreasing
  the fidelity of the FRAP technique. Except for one, movement of chromatin --- the binding sites.
  This is not easy to observe on a conventional wide-field microscope, and even on a confocal
  microscope is hard to spot. This makes it very easy to go unnoticed. I have performed an iFRAP
  experiment which makes it easier to detect.
  
  While the chromatin movement will not be a problem when performing FRAP on slow exchanging
  proteins, the other issues I found might.
  
  %% FRAP is not goof for very long time frames. Reference papers where it was used incorrectly?
  %% see Tim's paper
  
  Still, alternative techniques might be used to measure dynamics of histone variants in
  live cells. Namely, single molecule tracking would be an ideal candidate provided access
  to the required equipment\addref[has Davide published already about his ``new microscope''
  to do this?].
  
  \todo[inline]{search more for histone and single molecule tracking and imaging}
  
  Single-molecule imaging of histones for short period of times in live cells
  has recently been reported using super-resolution imaging\addref[nature methods 7(9):717-719,
  2010 and nature methods 8(1):7-9, 2011].
   
  Also, use of PA--GFP has been used to measure dynamics of H4 over \SI{90}{\ms} reporting
  differences between interphase chromatin and mitotic chromosomes\addref[Saera Hihara et al 2012].
  However, the difference between these two phases is the highest and might not be comparable to
  the difference between histones variants\todo{study this. Someone must have measured this}.
  
  %% did not mention if FRAP could have been used with H2A and H2B since these move faster after
  %% all. However, the ones really important on the nucleosome structure seem to be H3 and H4, and are
  %% the ones of more interest for us.
